\chapter{Descrizione Progetto}
\label{ch:DescrizioneProgetto}
Il progetto prevede la realizzazione di una piattaforma e-commerce per il negozio "Wardrobe".
Questa nuova versione del negozio di vestiti nasce con l'obiettivo di dare la possibilità ai clienti di acquistare i prodotti online e, quindi, ampliare il suo mercato.\\
La homepage del sito sarà progettata per mettere in evidenza le categorie dei prodotti che tratta il negozio come: Uomo, Donna e Bambini.\\
Il catalogo dei prodotti è mostrato in una pagina apposita dove il cliente può raffinare la ricerca del prodotto utilizzando una barra di ricerca e specifici filtri relativi alla categoria, marca, colore e taglia desiderata.\\
Ciascun prodotto ha una pagina dedicata che presenta in modo dettagliato tutte le sue informazioni quali: titolo, descrizione, prezzo ed eventuale sconto, immagini, colore e taglia con le possibili alternative e le recensioni dei clienti. Inoltre, da questa pagina, l'utente potrà scrivere la propria recensione del prodotto e aggiungere il prodotto al carrello.\\
La pagina dedicata al carrello permette ai clienti di vedere le informazioni dei prodotti desiderati, quali: immagine principale, titolo, descrizione, quantità e prezzo, oltre a mostrare il prezzo totale del carrello. Da questa pagina l'utente potrà iniziare il processo di acquisto.\\
Per poter acquistare i prodotti nel carrello il cliente deve inserire le informazioni di spedizione e la modalità di pagamento che vuole utilizzare.\\
Il cliente ha la possibilità di iscriversi alla piattaforma. L'iscrizione permette al cliente di effettuare ordini e scrivere recensioni, oltre a vedere lo storico degli acquisti. Il cliente iscritto ha la possibilità di modificare le informazioni del proprio account.\\
I dipendenti possono monitorare la piattaforma tramite un pannello di controllo. E' possibile aggiungere prodotti in vendita, aggiornare le informazioni di un prodotto o rimuovere un prodotto dalla piattaforma così da non renderlo visibile agli utenti ed infine aggiungere, modificare e/o eliminare specifiche categorie di prodotti.\\