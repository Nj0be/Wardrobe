chiedere a corradini il dubbio riguardo le postcondizioni
chiedere a corradini il dubbio con extends dei diagrammi dei casi d'uso.
enterprise:

dipendentemente dagli attributi che le classi di analisi hanno, aggiungere le associazioni con le classi

avere classi come attributi significa associazioni invece le classi che creano qualche istanza significa avere dipendenza, ad esempio i controller hanno dipendenze, ad esempio controllerOrder ha una dipendenza "use" con order, quindi:

. aggiungere le dipendenze use ai controllers
. aggiungere le associazioni con le classi i cui attributi presentano classi
. modificare gli stati dei diagrammi delle macchine a stati, aggiungendo stati per ogni attributo che cambia basandosi sul fatto che: "uno stato si riferisce all'insieme dei valori assunti dall'entità", quindi ad esempio ci dovrebbe essere uno stato per ogni attributo che viene modificato, es. siccome vi è un attributo "quantità" per un prodotto nel carrello, nel caso venisse modificata la quantità di un prodotto nel carrello, allora ci dovrà essere uno stato "Quantity modified".

. switchare ordersController con productsDiagram 


correggere da payed a paid.

"A usage is a dependency relationship in which one element (client) requires another element (or set of elements) (supplier) for its full implementation or operation."
correzioni da effettuare:
- ordersController ha una dipendenza "use" con order, orderItem, returnItem, province, User
- accountsController ha una dipendenza "use" con User, order
- cartController ha una associazione con cart, una dipendenza con cartItem, ProductVariant, User
- productsController ha una dipendenza "use" con Product, productVariant, size, color, ProductColor, brand, review, category


MACCHINE A STATI:
Cart Item: attributi: quantity, isActive
rimuovere lo stato "added to cart"
aggiungere check quantità:
 - initial
 - "attivo" 
    - se seleziono inattivo -> "inattivo"
        - se seleziono attivo -> "attivo"
    - stanghetta di riunione con il prossimo punto
 - se seleziono rimuovi prodotto -> X eliminazione
 - se la quantità è modificata -> "quantity modified" e si ricollega a sopra l if di attivo e inattivo

Customer Account: 
modificare che se l'utente non è modificato allora la linea si deve collegare a noT verified

Order:
Cambiare nome da "NotPayed" a "NotPaid"

ProductVariant:
-initial
-"active"
- se stock > 0 -> "Available"
- else -> "NonDisponibile"
- if quantità modificata -> "quantityModified" che si ricollega a active





togli dalla frase "Il sistema aggiunge il colore al sito" "il sito"


THE END
siamo arrivati a pagina 46 

DONE - correggere il diagramma dei componenti
DONE - correggere diagramma di deployment ponendo ai container <<container>> invece di execEnv, rimuovere repo +link, collegare i container con frecce "use".
DONE - porre i test case sottoforma di codice e non screenshots.
DONE - Refreshare il diagramma di deployment


THE END
siamo arrivati a pagina 56 